\begin{center}
  \Large
  \textbf{KATA PENGANTAR}
\end{center}

\addcontentsline{toc}{chapter}{KATA PENGANTAR}

\vspace{2ex}

% Ubah paragraf-paragraf berikut dengan isi dari kata pengantar

Puji dan syukur kehadirat Allah Swt. atas segala limpahan
berkah, rahmat, serta hidayah-Nya, penulis dapat menyelesaikan
penelitian ini dengan judul MENGHITUNG LUAS BANGUN DATAR PADA PAPAN TULIS MENGGUNAKAN YOLO.

Penelitian ini disusun dalam rangka pemenuhan bidang riset
di Departemen Teknik Komputer, serta digunakan sebagai persyaratan menyelesaikan pendidikan S1. Penelitian ini dapat terselesaikan tidak lepas dari bantuan berbagai pihak. Oleh karena itu,
penulis mengucapkan terima kasih kepada:

\begin{enumerate}[nolistsep]

  \item Keluarga, Ibu, Bapak dan Kakak-Kakak tercinta yang telah
  memberikan dorongan spiritual dan material dalam penyelesaian buku penelitian ini

  \item Bapak Dr. Supeno Mardi Susiki Nugroho, ST., MT. selaku Kepala Departemen Teknik Komputer, Fakultas Teknologi
  Elektro dan Informatika Cerdas (FTEIC), Institut Teknologi
  Sepuluh Nopember.

  \item Bapak Dr. Eko Mulyanto Yuniarno ST., MT. selaku
  dosen pembimbing I dan Bapak Dr. Supeno Mardi Susiki Nugroho, S.T., M.T. selaku dosen pembimbing II yang selalu memberikan arahan selama mengerjakan penelitian tugas akhir ini.
  
  \item Bapak-ibu dosen pengajar Departemen Teknik Komputer, atas
  pengajaran, bimbingan, serta perhatian yang diberikan kepada penulis selama ini.
  
  \item Seluruh teman-teman dari angkatan e58, Teknik Komputer,
  Laboratorium B401 Komputasi Multimedia, dan B201 Telematika Teknik Komputer ITS.

\end{enumerate}

Kesempurnaan hanya milik Allah SWT, untuk itu penulis memohon segenap kritik dan saran yang membangun. Semoga penelitian ini dapat memberikan manfaat bagi kita semua. Amin.

\begin{flushright}
  \begin{tabular}[b]{c}
    % Ubah kalimat berikut dengan tempat, bulan, dan tahun penulisan
    Surabaya, Mei 2022\\
    \\
    \\
    \\
    \\
    % Ubah kalimat berikut dengan nama mahasiswa
    Alan Luthfi
  \end{tabular}
\end{flushright}
