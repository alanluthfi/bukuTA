\chapter{PENDAHULUAN}
\label{chap:pendahuluan}
\section{Latar Belakang}
\label{sec:latarbelakang}

Alat pengajaran revolusioner pertama yaitu papan tulis hitam digunakan pada pengajaran dalam ruang kelas pada tahun 1801 dan memiliki dampak yang besar dalam pengajaran selama 200 tahun kedepan. Papan tulis pintar memiliki potensial untuk menjadi alat pengajaran revolusioner kedua. Seperti halnya papan tulis hitam yang menjadi bagian dari kunci ruang kelas pada abad sembilan belas dan abad dua puluh, papan tulis pintar memiliki kapabilitas untuk menjadi bagian dari kunci ruang kelas digital pada abad dua puluh satu. Meskipun relatif baru, papan tulis pintar memiliki kapasitas yang sama untuk merubah fundamental dan merevolusionerkan cara mengajar.
Dalam hal yang sama pada papan tulis hitam di zaman lampau yang merupakan teknologi yang digunakan oleh sekolah tradisional, papan tulis pintar sudah menampakkan fasilitas yang bisa digunakan oleh sekolah digital. Karena kapasitas papan tulis pintar yang bisa disematkan dalam ruang kelas modern, papan tulis pintar bisa menjadi katalis yang menggerakkan sekolah dari model tradisional berbasis kertas ke arah mode operasi digital yang lebih terintegrasi. Model sekolah tradisional berbasis kertas sudah ada dalam waktu yang cukup lama, namun kita mulai melihat pergantian pada sekolah diseluruh dunia untuk memaksimalkan potensial pembelajaran digital dan memanfaatkan keuntungan daripada kesempatan evolusi edukasi yang dibawa oleh dunia digital.
Namun perlu diingat bahwa ini adalah permulaan daripada revolusi. Tantangan yang dihadapi oleh guru dalam pengembangan pada ruang kelas digital adalah untuk melihat potensial yang tersedia lalu memanfaatkannya, dan berkolaborasi dengan rekan kerja maupun peserta didik untuk menggunakan alat pembelajaran dalam dunia digital secara efektif.  \citep{Lant2016}

Dengan menggunakan papan tulis pintar sebagai media pengajaran akan menyebabkan pembelajaran menjadi lebih menyenangkan, kreatif, dan menarik. papan tulis pintar mempengaruhi pembelajaran dalam berbagai cara. papan tulis pintar dapat meningkatkan keterlibatan pelajar saat di dalam kelas dan memotivasi pelajar untuk antusias dalam belajar. papan tulis pintar dapat membantu dalam pembelajaran dan bisa digunakan dalam berbagai lingkungan belajar \citep{jelyani_janfaza_soori_2014}

\section{Permasalahan}
\label{sec:permasalahan}

Papan tulis pintar yang dimana pada penerapannya butuh teknologi untuk medeteksi gambar bangun datar beserta parameternya lalu menghitung luas bangun datar pada papan tulis pintar.

\section{Batasan Masalah}
\label{sec:batasanmasalah}

Batasan-batasan permasalahan pada penelitian ini meliputi:
\begin{itemize}
	\item  bangun dattar apa saja yang bisa dihitung luasnya, yaitu persegi, persegi panjang, lingkaran, segitiga, trapesium.
	\item parameter angka harus dalam bilangan bulat.
	\item parameter angka hanya bisa dalam 1 digit
	\item penulisan parameter huruf dan angka harus diletakkan dibawah gambar bangun datar
\end{itemize}

\section{Tujuan}
\label{sec:Tujuan}

Tujuan dari penelitian ini adalah membuat program yang dapat mendeteksi bangun datar dan parameternya lalu menghitung luas bangun datar pada papan tulis menggunakan YOLO.

\section{Manfaat}
\label{sec:manfaat}

Manfaat dari penelitian ini adalah mempermudah proses belajar mengajar untuk materi perhitungan luas bangun datar pada papan tulis.