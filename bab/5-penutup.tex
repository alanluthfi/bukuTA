\chapter{PENUTUP}
\label{chap:penutup}
\section{Kesimpulan}
\label{sec:kesimpulan}
Berdasarkan hasil pengujian yang telah dilakukan, dapat ditarik beberapa kesimpulan sebagai berikut:
\begin{enumerate}
	\item Pendeteksian bangun datar beserta parameter huruf dan angka pada tulisan sendiri untuk beberapa bangun datar yang berhasil akan memberikan hasil perhitungan luas dengan benar.
	\item Pendeteksian bangun datar beserta parameter huruf dan angka pada tulisan sendiri untuk beberapa bangun datar masih kurang berhasil, dan perhitungan yang dihasilkan menjadi tidak didapatkan.
	\item Pendeteksian bangun datar beserta parameter huruf dan angka pada tulisan orang lain untuk semua bangun datar masih kurang berhasil, dan perhitungan yang dihasilkan menjadi tidak didapatkan.
	\item Pendeteksian bangun datar beserta parameter huruf dan angka pada tulisan orang lain untuk semua bangun bahkan tidak terdeteksi bangun datarnya, mengakibatkan hasil menjadi "..."
	\item Pendeteksian bangun datar beserta parameter huruf dan angka untuk semua bangun pada jarak yang berbeda akan berpengaruh pada hasil yang dilekuarkan. semakin jauh jarak maka akurasi akan semakin menurun.
	\item Jarak minimum agar pendeteksian bangun datar beserta parameter huruf dan angka berhasil didapatkan pada jarak pengambilan gambar 40cm dan maksimum 60cm
	\item Pendeteksian bangun datar beserta parameter huruf dan angka pada bangun datar yang gambarnya diarsir masih kurang berhasil, dan perhitungan yang dihasilkan menjadi tidak didapatkan.
\end{enumerate}

\section{Saran}
Untuk pengembangan lebih lanjut pada penelitian tugas akhir
ini, terdapat beberapa beberapa saran yang dapat dilakukan, antara lain:
\begin{enumerate}
	\item anotasi untuk objek yang belum terdeteksi perlu ditambahkan.
	\item dataset tulisan orang lain dibutuhkan agar menggunakan tulisan orang lain dapat dideteksi.
\end{enumerate}



