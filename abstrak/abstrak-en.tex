\begin{center}
  \large\textbf{ABSTRACT}
\end{center}

\addcontentsline{toc}{chapter}{ABSTRACT}

\vspace{2ex}

\begingroup
  % Menghilangkan padding
  \setlength{\tabcolsep}{0pt}

  \noindent
  \begin{tabularx}{\textwidth}{l >{\centering}m{3em} X}
    % Ubah kalimat berikut dengan nama mahasiswa
    \emph{Name}     &:& Alan Luthfi \\

    % Ubah kalimat berikut dengan judul tugas akhir dalam Bahasa Inggris
    \emph{Title}    &:& \emph{CALCULATING THE AREA OF BASIC SHAPES ON A WHITEBOARD USING YOLO} \\

    % Ubah kalimat-kalimat berikut dengan nama-nama dosen pembimbing
    \emph{Advisors} &:& 1. Dr. Eko Mulyanto Yuniarno S.T., M.T. \\
                    & & 2. Dr. Supeno Mardi Susiki Nugroho, S.T., M.T. \\
  \end{tabularx}
\endgroup

% Ubah paragraf berikut dengan abstrak dari tugas akhir dalam Bahasa Inggris
\emph{Smart whiteboards have the potential to become the second revolutionary learning tool
	after the traditional black whiteboard, because of the smart embeddable whiteboard in
	Modern classrooms can move schools towards a more integrated digital mode of operation. The smart whiteboard must have features that can distinguish the smart whiteboard
	with ordinary whiteboards, because smart whiteboards have more features or uses
	superior to ordinary whiteboards. Therefore, it is necessary to develop features on smart whiteboards. The purpose of the research is to create a program that can detect wakes
	plane and its parameters and then calculate the area of the flat shape on the smart whiteboard. Method
	which will be used is to use YOLO as a framework for
	making a flat wake detection program and its parameters.}

% Ubah kata-kata berikut dengan kata kunci dari tugas akhir dalam Bahasa Inggris
\emph{Keywords}: \emph{White Board}, \emph{Image Detection}, \emph{YOLO}.
