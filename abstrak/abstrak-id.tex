\begin{center}
  \large\textbf{ABSTRAK}
\end{center}

\addcontentsline{toc}{chapter}{ABSTRAK}

\vspace{2ex}

\begingroup
  % Menghilangkan padding
  \setlength{\tabcolsep}{0pt}

  \noindent
  \begin{tabularx}{\textwidth}{l >{\centering}m{2em} X}
    % Ubah kalimat berikut dengan nama mahasiswa
    Nama Mahasiswa    &:& Alan Luthfi \\

    % Ubah kalimat berikut dengan judul tugas akhir
    Judul Tugas Akhir &:&	{MENGHITUNG LUAS BANGUN DATAR PADA PAPAN TULIS MENGGUNAKAN YOLO} \\

    % Ubah kalimat-kalimat berikut dengan nama-nama dosen pembimbing
    Pembimbing        &:& 1. Dr. Eko Mulyanto Yuniarno ST., MT. \\
                      & & 2. Dr. Supeno Mardi Susiki Nugroho, S.T., M.T. \\
  \end{tabularx}
\endgroup

% Ubah paragraf berikut dengan abstrak dari tugas akhir
Papan tulis pintar memiliki potensial untuk menjadi alat pembelajaran revolusioner kedua
setelah papan tulis hitam tradisional, karena papan tulis pintar yang bisa disematkan dalam
ruang kelas modern bisa menggerakan sekolah ke arah mode operasi digital yang lebih terintegrasi. Pada papan tulis pintar harus memiliki fitur yang dapat membedakan papan tulis pintar
dengan papan tulis biasa, karena papan tulis pintar memiliki fitur-fitur atau kegunaan lebih
superior daripada papan tulis biasa. Oleh karena itu diperlukan pengembangan fitur pada papan tulis pintar. Tujuan penelitian adaah membuat program yang dapat mendeteksi bangun
datar dan parameternya lalu menghitung luas bangun datar pada papan tulis pintar. Metode
yang akan digunakan adalah dengan menggunakan YOLO sebagai framework pengerjaan dalam
pembuatan program deteksi bangun datar dan parameternya.

% Ubah kata-kata berikut dengan kata kunci dari tugas akhir
Kata Kunci: Papan tulis, deteksi gambar, YOLO.
